\cleardoublepage
\chapter{Methodik}

%%=========================================
\section{Anforderungen an das Programm}
\label{sec:programm}

Link zu dem github Projekt:
%GRAFIK PROGRAMM FLOW
Programmiersprache
Virtual Machine : Google Cloud Platform
Das Kommandozeilen Tool wurde in der Programmiersprache Java geschrieben 
%(vorteil Java: einmal kompiliert und dann fertig). 
Das Tool muss auf Linux Systemen funktionieren und soll dabei alle relevanten Hardware-Informationen des Computers auslesen, diese dann auf einer externen Datenbank speichern. %(MikroBenchmark Test?) 
Gleichzeitig soll anhand der Hardware-Informationen weitere relevante Benchmark Werte aus Hardware Datenbanken ausgelesen und ebenfalls gespeichert werden. Danach soll ein Ergebnis berechnet werden, wie schnell der Computer im Vergleich zu einem Referenz Computer performed. Es soll außerdem möglich sein, aus der Datenbank ein Gruppierung der vorhandenen Hardware Daten zu erstellen.
Für das Auslesen der Hardware Information wurden die Standard Kommandozeilen Befehle lscpu und xyz genutzt.
Als externe Datenbank zum speichern der relevanten Hardware Informationen wurde Postgresql gewählt. %(weil)
Die Datenbank lauft extern auf einer virtuellen Maschine, welche von der Google Cloud Platform bereit gestellt wurde. Somit ist es möglich das Programm überall ausführen können um die Daten zu speichern. Als Hilfstool zum bearbeiten der Datenbank wurde DBeaver genutzt. %(Datenbankschema Grafik)
Im Internet existieren eine Vielzahl von Online-Datenbanken für Hardware Informationen. Allerdings sind die meisten spezialisiert auf die Performance von Computerspielen und daher auch sehr Grafikkarten lastig. Hinzu kommt, dass keine einzige gefundene Online-Datenbank eine Programmierschnittstelle anbietet. Daher wurde die HtmlUnit Bibliothek für Java benutzt zum sogenannten scraping von Webseiten. Dies bedeutet, dass auslesen des gesamten Html Textes von Webseiten. Mithilfe des Document Object Model (kurz DOM) ist es möglich, direkt Elemente aus dem Html Text, wie beispielsweise Tabellen direkt anzusprechen und zu verarbeiten. %(Erklären) 

Es wurden zur Testen über 10 Computer genutzt. Zwar bietet Google Cloud Platform auch virtuelle Maschinen an, das Problem bei diesen ist allerdings, dass durch die virtualisierung keine standardmäßigen Hardware Komponenten genutzt werden, sondern für virtualisierung prozesse angepasste Komponenten die von den Hardware Herstellern extra entwickelt worden sind und dementsprechend auch in kaum einer Online-Datenbank zu finden sind. Hierfür wurden dennoch interne Mikrobenchmarking Tests gemacht.

http://cpudb.stanford.edu/processors : wissenschaftliche Datenbank für CPUs, allerdings SPEC Daten von 2006, sprich viele CPUs Benchmarks gar nicht vorhanden.


%%=========================================
\section{Online-Datenbanken}
\label{sec:database}

%%=========================================
\subsection{cpu-world}
\label{subsec:cpuworld}

%%=========================================
\subsection{userBenchmark}
\label{subsec:userbenchmark}

%%=========================================
\subsection{openBenchmarking}
\label{subsec:openbenchmarking}



%%=========================================
\section{Berechnung des Scores}
\label{sec:score}